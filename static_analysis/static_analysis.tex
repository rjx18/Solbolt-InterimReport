\chapter{Gas bounds analysis}
\label{chap:gasbounds}

Now that we understand the basics of the Ethereum blockchain as well as 
the generation of EVM instructions from Solidity code, let us now examine
how we can derive gas estimates from it. Although many 
of the opcodes have a fixed gas cost, a large part of gas used still may be derived from 
storage and memory costs, which require more detailed analysis. Therefore, a
naive implementation of simply adding up the fixed costs of each instruction
will not be sufficiently precise or sound for our analysis.

To solve this, we propose making use of and extending Mythril, a symbolic execution engine for EVM bytecode,
in order accurately infer gas bounds for each line of Solidity code. We will discuss the steps involved in detail within
the following sections.

\section{Construction of basic blocks}

After the generation of the EVM bytecode, we must first define the notion of a basic blocks, 
which are the maximal straight-line sequence of consecutive instructions where there are no branches
in apart from the first instruction, and no branches out apart from the last instruction 
\cite{HennessyPatterson12}. For EVM instructions, this can be defined as such \cite{constructcfg}:

\begin{definition}[basic blocks]

  \textit{Given an EVM programme $P \equiv b_0,...,b_p$, we have}:
  \[
    blocks(P) \equiv \Bigg\{ B_i \equiv b_i,...,b_j \:\Bigg|\:
    \begin{aligned}
      & (\forall k.i < k < j, b_k \notin Jump \cup End \cup \{ \textit{\texttt{JUMPDEST}} \}) \: \land  \\
      & (i = 0 \lor b_i \equiv \textit{\texttt{JUMPDEST}} \lor b_{i-1} \equiv \textit{\texttt{JUMPI}}) \: \land \\
      & (j = p \lor b_j \in Jump \lor b_j \in End \lor b_{j+1} \equiv \textit{\texttt{JUMPDEST}})
    \end{aligned}
    \:
    \Bigg\}
  \]
  \textit{where}
  \[
    \begin{aligned}
    &Jump \equiv \{ \textit{\texttt{JUMP}}, \textit{\texttt{JUMPI}} \} \\
    &End \equiv \{ \textit{\texttt{REVERT}}, \textit{\texttt{STOP}}, \textit{\texttt{INVALID}} \}
    \end{aligned}
  \]

\end{definition}

The generation of basic blocks can therefore be done in linear time, by parsing each
EVM instruction line by line. After this, we can also then generate the set of valid jump addresses,
which are simply the set of addresses with \textit{\texttt{JUMPDEST}} instructions.

\section{Symbolic execution}

Next, in order to understand the maximum gas costs incurred by each basic block during transaction
execution, we can use symbolic execution to explore as many of the possible execution branches that
can be reached. This requires a detailed model of the internals of the EVM, such as how it
handles the stack, and fortunately Mythril is a tool that has such capabilities. Mythril \cite{mythril} is originally
designed by ConsenSys for verification of smart contracts and analysing which path conditions are able to
reach certain states to induce undesired behaviour, such as killing a smart contract. However, its \texttt{LASER} symbolic
execution engine can also be used for our purposes of gas estimation. It also has the ability to output a control flow graph
of the paths it took during symbolic execution. 

To do this, we propose extending its gas meter module to also keep track of the total gas used by each basic block. Then,
we are able to output a control flow graph with the corresponding gas usage information, and then calculate the maximum 
and minimum gas used by each block out of all the possible paths.

There are also certain tuneable hyperparameters that Mythril offers, such as a bound on the number of times a loop can be explored,
as well as setting the states of storage variables in a smart contract and the global blockchain state. We would like to also expose
these settings for the user, and allow a user to define the intial state used for symbolic execution to obtain a more accurate
estimate.

There are also other tools such as Oyente \cite{oyente} (and EthIR, an extension of it,) that are able to perform symbolic execution and produce a 
control flow graph. However, when testing their capabilities, it was found that they were no longer actively maintained,
and contained many errors that required patching before they worked. They also did not support the latest Solidity version, and were
missing numerous newly introduced opcodes, such as SHR (shift right) and SHL (shift left).

\section{Loop bounds inference}

Finally, when symbolically executing a smart contract, we will likely run into the problem of path explosion. For large and complex smart
contracts, the number of possible execution paths will grow exponentially with an increase in bytecode size. However, smart contracts
are typically much simpler compared to executable desktop programmes, because of the gas limit imposed by each block. They are also guaranteed
to halt, either by successfully completing the execution, or by running out of gas. Even so, the number of paths that may need to be traversed
before hitting the block gas limit may still be intractible, and require some heuristics for optimisation. Therefore, we propose to infer
loop boundaries for loop patterns that can be evaluated statically, by extending the Solidity compiler that traverses the abstract syntax tree of the 
Yul intermediate representation. To our knowledge, no other existing tool performs this kind of
static analysis on Yul IR.

The method chosen is adapted from Lokuciejewski et al \cite{loopanalysis}, where they describe a non-iterative approach to analyse loop
boundaries of certain loops that match their constraints. These constraints are that the exit conditions must either depend on a constant, 
non-modified variable within the loop, or at most a single modified variable, and that all condition statements are affine expressions of variables.
We can adopt a similar method for loops in the Yul language.

First, we take advantage of the fact that all computation on the EVM is deterministic. Therefore, any variable must either have been declared statically
within the source code, or depends on a set of operations that acts on a finite set of parameters, such as the function arguments, or global parameters
within the blockchain, such as the block number and timestamp. For our analysis, we transform any calls to external contracts into a parameter as well.

Then, we can define the following relationship for each variable $V$, and each time $n$ it is newly assigned to 
(if $n$ is omitted, assume we are retrieving the last assignment):

\begin{dmath}
  DerivesFrom(V: \texttt{Var}, n: \texttt{Const?}): \texttt{Expr} \equiv
  \begin{cases}
    \texttt{Unknown} \\
    \texttt{Const} \\
    \texttt{Var}, n'   \\
    Induct(\texttt{Var}) \\
    BinOp(left: \texttt{Expr},right: \texttt{Expr}) \\
    If(cond: \texttt{Expr}, branchTrue: \texttt{Expr}, branchFalse: \texttt{Expr})
    % Loop(start: \texttt{Expr}, op: \texttt{Expr}, numLoops: \texttt{Expr})
  \end{cases}
\end{dmath}

This relationship helps us identify how a variable is calculated at each point in the programme. Next,
for each expression in a Yul programme, we also have the following operator $Op$:

\addfunctions{Let}

\begin{dmath}
  Op(e: \texttt{Expr}): \texttt{Expr} \equiv
  \begin{cases}
    \texttt{Const} & \text{if $e$ is type \texttt{Const}} \\
    e & \text{if $e$ is type \texttt{Var} and $DerivesFrom(e)$ does not exist} \\
    DerivesFrom(e) & \text{if $e$ is type \texttt{Var} and $DerivesFrom(e)$ exists} \\
    DerivesFrom(retVar) & \text{if $e$ is $Func(Args)$ returns $retVar$} \\
    BinOp(Op(left), Op(right)) & \text{if $e$ is $BinOp(left, right)$} \\
    newExtVar & \text{if $e$ is a call to an external function}
  \end{cases}
\end{dmath}

This relationship applies and simplifies each operation into the simplest $DerivesFrom$ relationship possible.
Then, we walk the abstract syntax tree for the Yul IR using the pseudocode in the following subsections, 
which can be done at the same time as EVM code compilation.

\subsection{Assignment statements}

For assignment statements, we simply call the $Op$ operator on the expression,
in order to simplify it into a $DerivesFrom$ relationship of constants, operators and 
parameters, which we push to the current $DerivesFrom$ data structure.

\begin{sflisting}
  procedure Walk(DerivesFrom, [s : s_tail], LoopConstraints)
    switch s
      case s matches (LET v := expr)
        DerivesFrom.push(v, Op(expr))
        Walk(DerivesFrom, Conds, s_tail, LoopConstraints)
  end procedure 
\end{sflisting}

For example, say we have the following Yul function:
\begin{sflisting}
  function fun_simpleLoop_39(var_argument_11) {
    let _1 := var_argument_11
    let expr_16 := _1
    let expr_17 := 0x01
    let expr_18 := add(expr16, expr_17)
}
\end{sflisting}

Then, we obtain the following $DerivesFrom$ relationships:
\[
  \begin{aligned}
    DerivesFrom(\texttt{_1}) &\equiv Op(\texttt{var_argument_11}) \\
                             &\equiv \texttt{var_argument_11} \\
  \end{aligned}
\]
\[
  \begin{aligned}
    DerivesFrom(\texttt{expr_16}) &\equiv Op(\texttt{_1}) \\
                             &\equiv DerivesFrom(\texttt{_1}) \\
                             &\equiv \texttt{var_argument_11} \\
  \end{aligned}
\]
\[
  \begin{aligned}
    DerivesFrom(\texttt{expr_17}) &\equiv Op(\texttt{0x01}) \\
                             &\equiv \texttt{0x01} \\
  \end{aligned}
\]
\[
  \begin{aligned}
    DerivesFrom(\texttt{expr_18}) &\equiv Op(add(\texttt{expr16}, \texttt{expr_17})) \\
                             &\equiv add(Op(\texttt{expr16}), Op(\texttt{expr_17})) \\
                             &\equiv add(DerivesFrom(\texttt{expr_16}), DerivesFrom(\texttt{expr_17})) \\
                             &\equiv add(\texttt{var_argument_11}, \texttt{0x01}) \\
  \end{aligned}
\]

\subsection{If statements}

For if statements, we first simplify the condition using the $Op$ operator, and then clone the
current $DerivesFrom$ struct. We then use the cloned struct to walk the true branch of the if statement.
In Yul, there are no else statements, so control flow returns to the original programme point before the branch.

Then, we iterate through the variables within the $DerivesFrom$ struct of the upper scope, and then check if
they have been modified by the if branch. If they have, we push an If expression as the variable's $DerivedFrom$
relationship, with the simplified condition obtained before.

\begin{sflisting}
  procedure Walk(DerivesFrom, [s : s_tail], LoopConstraints)
    switch s
      case s matches (IF (c) {if_stats})
        // Create new scope
        DerivesFrom_ = DerivesFrom.clone()
        c_ = Op(c)
        Walk(DerivesFrom_, if_stats)

        // For variables in upper scope
        for v in DerivesFrom do
          v_prev = DerivesFrom.pop(v)
          v_if = DerivesFrom_.get(v)
          if (v_prev != v_if)
            DerivesFrom.push(v, If(c_, v_if, v_prev))
        Walk(DerivesFrom, Conds, s_tail, LoopConstraints)
  end procedure 
\end{sflisting}

\subsection{For loop statements}

Finally, for loops, we have to first check if certain constraints are met before we are able to analyse it statically.
The constraints we have for loops that can be statically analysed are:
\begin{enumerate}
  \item There is only one induction variable for each loop.
  \item The induction variable can only be initialised to a constant variable, or an affine expression of another previous
        induction variable from a previous loop
  \item The loop condition can only depend on this induction variable and other constants or variables that remain unchanged
  \item The induction variable can only be modified within the post-loop header, and not anywhere else in the body
  \item The induction variable must be modified by a constant or a variable that remains unchanged
\end{enumerate}

To enforce these constraints, we first obtain a set of all variables that are modified within the pre- and post-loop
headers, as well as the loop body. We check that any variables set in the pre-loop header are not \texttt{Unknown} (which means they
were modified within a previous outer loop), and that they are either constant or an affine expression of an $Induction(\texttt{Var})$
from an outer loop.

Then, we obtain a set of all loop conditions, by first including the condition
within the loop header, and then traversing the loop body to find any break statements, and then including the
conditions needed to reach it. After that, we extract all the variables involved in the loop conditions.

Next, we check for the intersection of variables modified in the pre- and post-loop headers and involved in
the loop conditions. There should only be one variable left, which will be our induction variable. We then check
that the induction variable as well as the other variables involved in the loop conditions are not modified anywhere 
else within the loop body.

If the constraints are satisfied, we can analyse the loop using the induction variable and the mutations performed on it,
described in a later section, to obtain a set of loop constraints. These are passed on when walking the inner loop body,
so that they can be composed with any nested loops. We then create a new $DerivesFrom$ struct for the inner loop scope, 
and push an \texttt{Unknown} value for the $DerivesFrom$ relationship of any variables modified within the loop. For the 
induction variable, we push a special $Induction(\texttt{Var})$ variable to note that it is an induction variable. We now walk
the inner loop bodies in the same way.

If the constraints are not satisfied, we simply exit the loop altogether, because any nested loops will also not be analysable.
In these cases, we will need to fallback into using symbolic execution with a bound on loop unfolding.

Finally, when exiting the loop, we append fresh variables to any of the variables that were modified within the loop. This
may be improved on in the future for variables that meet certain constraints, where we can statically derive their final 
values as another $DerivesFrom$ relationship between the number of loop invocations and previously defined variables.

\begin{sflisting}
  procedure Walk(DerivesFrom, [s : s_tail], LoopConstraints)
    switch s
      case s matches (FOR (pre) (c) (post) (loop))
        // Checks if loop can be analysed
        canBeAnalysed = True

        loopHeaderVars = new Set()
        // Check loop header assignments
        // These must be either constant, or 
        // set to an affine expression of one induction variable
        DF_Pre = emptyDerivesFrom()
        Walk(DF_Pre, pre)
        for v in DF_Pre
          if (DF_Pre.get(v) is not Unknown and is affine)
            loopHeaderVars.insert(v)

        modifiedPostVariables = new Set()
        // Find possible induction variables in post
        DF_Post = emptyDerivesFrom()
        Walk(DF_Post, post)
        for v in DF_Post
          modifiedPostVariables.insert(v)

        loopCondsVariables = new Set()
        // Find loop conditions, 
        // and get the set of variables they derive from
        v, conds = findVars(c)
        v_, conds_ = findBreakCondVars(loop)
        loopConditions = unionConds(conds, conds_)
        loopCondsVariables.insert(v ++ v_)

        inductionVars = intersect(loopHeaderVars, modifiedPostVariables, loopCondsVariables)

        modifiedLoopVariables = new Set()
        // Find variables modified within a loop
        modifiedLoopVariables = findModifiedLoopVars(loop)

        // check if can be analysed
        if inductionVars.length != 1 or !disjoint(loopCondsVariables, modifiedLoopVariables)
          canBeAnalysed = False

        // If loop cannot be analysed, then any nested
        // loops also cannot be analysed, so we skip them
        if canBeAnalysed
          inductionVar = inductionVars[0]

          // Get loop constraints
          LoopConstraints_ = calculateLoopBounds(inductionVars[0], DF_Pre, DF_Post, LoopConstraints)
          logConstraints(LoopConstaints_)

          // Create new DerivesFrom and walk loop body
          // Induction variable is set to special value 
          // composition with nested loops
          DF_Loop = union(DF_Pre, DerivesFrom)

          DF_Loop.push(inductionVars[0], getNewInductionVar(inductionVars[0]))
          for v in modifiedLoopVariables
            DF_Loop.push(v, Unknown)
          
          Walk(DF_Loop, loop, LoopConstaints_)

        // Push new fresh variables for each variable modified 
        // in loop, continue with rest of control flow
        for v in modifiedLoopVariables
          DerivesFrom.push(v, getNewVar())
        
        Walk(DerivesFrom, Conds, s_tail, LoopConstraints)
  end procedure 
\end{sflisting}

Although these constraints seem rather strict, we observe that they are often satisfied by many well-formed Yul programmes.
This is partly because of the gas constraints when developing in Solidity, and therefore complex loops are not often used 
by developers.

\subsection{Loop constraint polytopes}

Next, in order to infer loop bounds as a parametric equation, the \texttt{calculateLoopBounds} will output a number of 
constraints in the form of a \textit{polyhedron}, which is an N-dimensional geometrical object defined by a set of linear
inequalities \cite{loopanalysis}:

\begin{definition}[polyhedrons]

  \textit{An $N$-dimensional polyhedron $P$ is defined by}:
  \[
    P := \{ x \in \mathbb{Z}^N | Ax = a, Bx \geq b \}
  \]

  \textit{where $A, B \in \in \mathbb{Z}^{m \times N}$ and $a, b \in \mathbb{Z}^m$ and $m \in \mathbb{N}$.}

\end{definition}

In our case, the constraints are also \textit{polytopes}, since we have $|P| < \infty$. To obtain this,
we construct the first inequality with the initial value the induction variable is set to, and check to see
if it is incremented or decremented in each loop body. Assuming it is set to $a$ initially and incremented each time, 
we have for induction variable $i$ that $i \geq a$. 

Then, for each of the loop conditions acting on $i$, we normalise them and derive another set of inequalities. For example,
if the loop continues on the condition that $i < b$, and breaks when $i == c$, we then have $i < b$, $i < c$ and $i > c$ as additional
inequalities. We can then find the smallest intersection between all of these inequalities, which is parameterised by $a$, $b$ and $c$,
and the loop bounds can be derived from there together with the amount that the induction variable is incremented or decremented by.
These parameters can then be defined by the user arbitrarily, and the loop bounds can be recalculated directly without needing to
re-compute the loop analysis.

In the case of nested loops, the previous loop constraits from the outer loop can be included together with the constraints of the
current loop. For example, say the loop described previously had an inner loop with the inequalities $j > 2 \times i + 1$, $j < d$. Then, we
can perform a similar intersection, this time in 2 dimensions instead of just 1, as visualised in Figure \ref{fig:polytopes}.
Each loop execution is also marked with a dot within the shaded intersection area.

\begin{figure}[!h]
  \centering
  \caption{Polytope calculated for a nested loop}
  \label{fig:polytopes}
  \begin{tikzpicture}
    \begin{axis}[
      xmin=-1, xmax=7,
      ymin=-1, ymax=12,
      axis lines=middle,
      xlabel=$i$,
      ylabel=$j$,
    ]  
    % \addplot[color=red]{15};
      \addplot +[red, mark=none, name path=A] coordinates {(4, -1) (4, 12)} node[right,pos=0.5] {$i < b$};
      \addplot +[red, mark=none, name path=B] coordinates {(6, -1) (6, 12)} node[right,pos=0.5] {$i < c$};
      \addplot +[red, mark=none, name path=C] coordinates {(1, -1) (1, 12)} node[left,pos=0.6] {$i \geq a$};
      \addplot +[blue, mark=none, name path=D] coordinates {(-1, -1) (6, 13)} node[right,pos=0.35] {$j \geq 2 \times i + 1$};
      \addplot +[blue, mark=none, name path=E] coordinates {(-1, 10) (7, 10)} node[above,pos=0.4] {$j < d$};
      \path[name path=xaxis] (\pgfkeysvalueof{/pgfplots/xmin}, 0) -- (\pgfkeysvalueof{/pgfplots/xmax},0);
      \addplot[gray, pattern=north west lines, opacity=0.5] fill between[of=D and E, soft clip={domain=1:4}];
      \addplot [only marks] table {
      1   3
      1   4
      1   5
      1   6
      1   7
      1   8
      1   9
      2   5
      2   6
      2   7
      2   8
      2   9
      3   7
      3   8
      3   9
      };
    \end{axis}
  \end{tikzpicture}
\end{figure}

This information can then be output as debug metadata in the final EVM assembly.
The final step involves making use of the mapping from Yul IR to the generated EVM assembly. 
We can modify the symbolic execution engine to read the debug metadata for each tag generated from the Yul IR.

Then, instead of relying on symbolic execution, if the user specified concrete values
for the paramters involved, we can calculate the loop bounds directly, and estimate the gas cost as the
cost for running the block once, multiplied by the number of times the loop is executed.

\subsection{Abstract Interpretation}

An alternative method we can use is Abstract Interpretation via the interval domain, similar to the original implementation in \cite{loopanalysis}.
This method iteratively calculates at every programme point, the interval of values each variable can take at each programme point.
Then, programme slicing and a similar polytope analysis is carried out for loops that meet certain constraints, before deriving the
final loop bounds. 

This method is more involved in that it requires a control flow graph to be produced before any analysis
can take place. In addition, it does not output a parametric solution, so therefore the analysis process needs to be repeated for 
each time the user specifies new values for the function parameters. However, it has the advantage that the user does not need to specify
concrete values, but can also specify a range of values each function parameter can take at the function entry point.

% \begin{lstlisting}
%   data Expr = 
%     int | 
%     Var | 
%     BinOp (Expr) (Expr) | 
%     Loop (Expr) (Expr) (Expr)

%   DerivesFrom(V, n) => (Expr, Cond)

% \end{lstlisting}

% Then, we can use the following pseudocode to calculate the DerivesFrom relationships for each variable, for Yul programme $P \equiv [s_0 : s_{tail}]$:

% \begin{lstlisting}
% eval(derives_from_struct, conditions, [s_0: s_tail], loop_break_conditions?) {
%   switch (s_0) {
%     case s_0 matches var := expr {
%       pushDerivesFrom(derives_from_struct, var, op(expr), conditions)
%       eval(derives_from_struct, conditions, s_tail)
%     }

%     case s_0 matches if (expr) {if_statements} {
%       conditions.push(op(expr))
%       eval(derives_from_struct, conditions, s_tail)
%       conditions.pop()
%     }

%     case s_0 matches for {pre_loop_statements} cond {post_loop_statments} {loop_statements} {
%       eval(derives_from_struct, conditions, pre_loop_statements)

%       D = emptyDerivesFrom()

%       # calculates what mutations are done on variables within a loop
%       mutations = []

%       loop_break_conditions = cond

%       conditions.push(cond)

%       # evaluate one invocation of a loop to find derives from relationships
%       eval(D, conditions, loop_statements)



%     }

%     case s_0 matches break {
%       assert (loop_break_conditions)

%       # if there is a break, we append the current list of conditions into the loop break conditions list
%       loop_break_conditions.push(conditions)
%     }
%   }
% }

% \end{lstlisting}

% DERIVES_FROM (var) => sequence of parameters it derives from

% eval(var := expr, statements) => DERIVES_FROM(var) = OP(expr), eval(statements)
% eval(mstore(addr, expr), statements) => DERIVES_FROM(m_addr_var) = OP(expr), eval(statements) // same for sstore

% eval(if (expr) [if_statements], statements) => IF(OP(expr), eval([if_statements]), eval(statements)

% eval(switch(expr)) => construct blocks for each branch, return to previous flow

% OP(func(expr,...)) = DERIVES_FROM(ret_var)
% OP(mload(addr)) = DERIVES_FROM(m_addr_var) // same for sload
% OP(sub(expr1, expr2)) = OP(expr1) - OP(expr2)

% OP(var) = DERIVES_FROM(var)
% OP(const) = const

% DERIVES_FROM(expr_30) = OP(lt(cleanup_t_uint256(expr_28), cleanup_t_uint256(expr_29)))
%   = OP(cleanup_t_uint256(expr_28)) < OP(cleanup_t_uint256(expr_29))
%   = OP(expr_28) < OP(expr_29)
%   = OP(_4) < OP(_5)
%   = OP(var_i_25) < OP(var_loopEndIndex_19)
%   = OP(var_i_25) < OP(expr_22)
%   = OP(var_i_25) < OP(expr_22)
%   = OP(var_i_25) < OP(checked_sub_t_uint256(expr_20, convert_t_rational_1_by_1_to_t_uint256(expr_21)))
%   = OP(var_i_25) < OP(expr_20) - OP(convert_t_rational_1_by_1_to_t_uint256(expr_21)))
%   = DERIVES_FROM(var_i_25) < OP(var_argument_15) - OP(expr_21))
%   = DERIVES_FROM(var_i_25) < OP(var_argument_15) - 0x01)
%   = FOR(var_i_25, OP(convert_t_rational_0_by_1_to_t_uint256(expr_26)), OP(_2)) < var_argument_15 - 0x01)
%   = FOR(var_i_25, OP(expr_26), OP(increment_t_uint256(_3))) < var_argument_15 - 0x01)
%   = FOR(var_i_25, 0x00, OP(increment_t_uint256(_3))) < var_argument_15 - 0x01)
%   = FOR(var_i_25, 0x00, OP(add(_3, 1))) < var_argument_15 - 0x01)
%   = FOR(var_i_25, 0x00, OP(_3) + OP(1)))) < var_argument_15 - 0x01)
%   = FOR(var_i_25, 0x00, OP(var_i_25) + 1))) < var_argument_15 - 0x01)

%   // until hit a cycle I guess?

% Then we know the number of times this will run =>
% var_argument_15 - 0x01 - (0 + n * 1) = 0
% var_argument_15 - 1 - n = 0
% n = var_argument_15 - 1