\chapter{Gas bounds analysis of EVM instructions}

Now that we understand the basics of the Ethereum blockchain as well as 
the generation of EVM instructions from Solidity code, let us now examine
how we can derive gas estimates from it. Although many 
of the opcodes have a fixed gas cost, a large part of gas used still may be derived from 
storage and memory costs, which require more detailed analysis. Therefore, a
naive implementation of simply adding up the fixed costs of each instruction
will not be sufficiently precise or sound for our analysis.

To solve this, we propose making use of and extending Mythril, a symbolic execution engine for EVM bytecode,
in order accurately infer gas bounds for each line of Solidity code. We will discuss the steps involved in detail within
the following sections.

\section{Construction of basic blocks}

After the generation of the EVM bytecode, we must first define the notion of a basic blocks, 
which are the maximal straight-line sequence of consecutive instructions where there are no branches
in apart from the first instruction, and no branches out apart from the last instruction 
\cite{HennessyPatterson12}. For EVM instructions, this can be defined as such \cite{constructcfg}:

\begin{definition}[basic blocks]

  \textit{Given an EVM program $P \equiv b_0,...,b_p$, we have}:
  \[
    blocks(P) \equiv \Bigg\{ B_i \equiv b_i,...,b_j \:\Bigg|\:
    \begin{aligned}
      & (\forall k.i < k < j, b_k \notin Jump \cup End \cup \{ \textit{\texttt{JUMPDEST}} \}) \: \land  \\
      & (i = 0 \lor b_i \equiv \textit{\texttt{JUMPDEST}} \lor b_{i-1} \equiv \textit{\texttt{JUMPI}}) \: \land \\
      & (j = p \lor b_j \in Jump \lor b_j \in End \lor b_{j+1} \equiv \textit{\texttt{JUMPDEST}})
    \end{aligned}
    \:
    \Bigg\}
  \]
  \textit{where}
  \[
    \begin{aligned}
    &Jump \equiv \{ \textit{\texttt{JUMP}}, \textit{\texttt{JUMPI}} \} \\
    &End \equiv \{ \textit{\texttt{REVERT}}, \textit{\texttt{STOP}}, \textit{\texttt{INVALID}} \}
    \end{aligned}
  \]

\end{definition}

The generation of basic blocks can therefore be done in linear time, by parsing each
EVM instruction line by line. After this, we can also then generate the set of valid jump addresses,
which are simply the set of addresses with \textit{\texttt{JUMPDEST}} instructions.

\section{Symbolic execution}

Next, in order to understand the maximum gas costs incurred by each basic block during transaction
execution, we can use symbolic execution to explore as many of the possible execution branches that
can be reached. This requires a detailed model of the internals of the EVM, such as how it
handles the stack, and fortunately Mythril is a tool that has such capabilities. Mythril \cite{mythril} is originally
designed by ConsenSys for verification of smart contracts and analysing which path conditions are able to
reach certain states to induce undesired behaviour, such as killing a smart contract. However, its \texttt{LASER} symbolic
execution engine can also be used for our purposes of gas estimation. It also has the ability to output a control flow graph
of the paths it took during symbolic execution. 

To do this, we propose extending its gas meter module to also keep track of the total gas used by each basic block. Then,
we are able to output a control flow graph with the corresponding gas usage information, and then calculate the maximum 
and minimum gas used by each block out of all the possible paths.

There are also certain tuneable hyperparameters that Mythril offers, such as a bound on the number of times a loop can be explored,
as well as setting the states of storage variables in a smart contract and the global blockchain state. We would like to also expose
these settings for the user, and allow a user to define the intial state used for symbolic execution to obtain a more accurate
estimate.

There are also other tools such as Oyente \cite{oyente} (and EthIR, an extension of it,) that are able to perform symbolic execution and produce a 
control flow graph. However, when testing their capabilities, it was found that they were no longer actively maintained,
and contained many errors that required patching before they worked. They also did not support the latest Solidity version, and were
missing numerous newly introduced opcodes, such as SHR (shift right) and SHL (shift left).

\section{Loop bounds inference}

Finally, when symbolically executing a smart contract, we will likely run into the problem of path explosion. For large and complex smart
contracts, the number of possible execution paths will grow exponentially with an increase in bytecode size. However, smart contracts
are typically much simpler compared to executable desktop programs, because of the gas limit imposed by each block. They are also guaranteed
to halt, either by successfully completing the execution, or by running out of gas. Even so, the number of paths that may need to be traversed
before hitting the block gas limit may still be intractible, and require some heuristics for optimisation. Therefore, we propose to infer
loop boundaries for simple loop patterns that can be evaluated statically, by extending the Solidity compiler that traverses the Yul intermediate
representation.

DERIVES_FROM (var) => sequence of parameters it derives from

var := expr => DERIVES_FROM(var) = OP(expr)
mstore(addr, expr) => DERIVES_FROM(m_addr_var) = OP(expr) // same for sstore

if (expr) [statements] => OP(expr) ? 

OP(func(expr,...)) = DERIVES_FROM(ret_var)
OP(mload(addr)) = DERIVES_FROM(m_addr_var) // same for sload
OP(sub(expr1, expr2)) = OP(expr1) - OP(expr2)

OP(var) = DERIVES_FROM(var)
OP(const) = const

DERIVES_FROM(expr_30) = OP(lt(cleanup_t_uint256(expr_28), cleanup_t_uint256(expr_29)))
  = OP(cleanup_t_uint256(expr_28)) < OP(cleanup_t_uint256(expr_29))
  = OP(expr_28) < OP(expr_29)
  = OP(_4) < OP(_5)
  = OP(var_i_25) < OP(var_loopEndIndex_19)
  = OP(var_i_25) < OP(expr_22)
  = OP(var_i_25) < OP(expr_22)
  = OP(var_i_25) < OP(checked_sub_t_uint256(expr_20, convert_t_rational_1_by_1_to_t_uint256(expr_21)))
  = OP(var_i_25) < OP(expr_20) - OP(convert_t_rational_1_by_1_to_t_uint256(expr_21)))
  = DERIVES_FROM(var_i_25) < OP(var_argument_15) - OP(expr_21))
  = DERIVES_FROM(var_i_25) < OP(var_argument_15) - 0x01)
  = FOR(var_i_25, OP(convert_t_rational_0_by_1_to_t_uint256(expr_26)), OP(_2)) < var_argument_15 - 0x01)
  = FOR(var_i_25, OP(expr_26), OP(increment_t_uint256(_3))) < var_argument_15 - 0x01)
  = FOR(var_i_25, 0x00, OP(increment_t_uint256(_3))) < var_argument_15 - 0x01)
  = FOR(var_i_25, 0x00, OP(add(_3, 1))) < var_argument_15 - 0x01)
  = FOR(var_i_25, 0x00, OP(_3) + OP(1)))) < var_argument_15 - 0x01)
  = FOR(var_i_25, 0x00, OP(var_i_25) + 1))) < var_argument_15 - 0x01)

  // until hit a cycle I guess?

Then we know the number of times this will run =>
var_argument_15 - 0x01 - (0 + n * 1) = 0
var_argument_15 - 1 - n = 0
n = var_argument_15 - 1