\chapter{Background}

\section{The Ethereum Blockchain}

The Ethereum Blockchain can be described as a decentralised, transaction-based 
state machine \cite{ethereumyellowpaper}. It runs on a proof-of-work system,
where, simply put, miners run a computationally difficult algorithm, called Ethash
\cite{ethash}, to find a valid nonce for a given block through trial and error. 
The difficulty of the algorithm is adjusted dynamically to target the mean block time, 
currently at around 13 seconds \cite{eip100}. Once a valid nonce is generated,
it is very easy for other clients to verify, but almost impossible to tamper with,
since changing even one transaction will lead to a completely different hash \cite{ethereumpow}.

Miners are rewarded in Ether with each block they mine, and this gives an economic incentive for more 
parties to participate in the network. Since the Ethash algorithm is being run by thousands of independent miners concurrently,
this makes the Ethereum blockchain more decentralised and secure. A malicious miner
will need to be able to consistently solve block nonces faster than other miners to 
include malicious transactions into each new block, or even reorder blocks \cite{eth51percentattack}. 
To do this, they would need to control over 51\% of the total hashing power 
of the network, which is incredibly costly and difficult to achieve for a highly decentralised
network. This therefore makes the Ethereum blockchain virtually tamper-proof.

The world state of the Ethereum blockchain is a mapping between 160-bit addresses 
and the corresponding account state. This world state can be altered through the 
execution of transactions, which represent valid state transitions. Formally:
\[ \sigma_{t + 1} \equiv \Upsilon(\sigma_{t}, T) \]
where is $\Upsilon$ the Ethereum state transition function, $T$ is a valid transaction,
and $\sigma_{t}$ and $\sigma_{t + 1}$ are neighbouring states \cite{ethereumyellowpaper}.
The main innovation of Ethereum is that $\Upsilon$ allows any arbitrary computation to
be executed, and $\sigma$ is able to store any arbitrary state, not limited to just account 
balances. This leads us to the concept of smart contracts, which are arbitrary code deployed
and stored on the Ethereum blockchain, that can be triggered by contract calls in the form
of transactions submitted to the blockchain.

\section{Smart contracts}

The idea of smart contracts was first introduced in 1994 by Nick Szabo \cite{nickszabosmartcontracts1},
where he describes it as a "computerised transaction protocol that executes the terms of a contract". 
Contractual agreements between parties can be directly embedded within the systems that we 
interact with, which are able to self-enforce the terms of the contract
in a way that "minimise[s] exceptions both malicious and accidental, and minimise[s] the need
for trusted intermediaries", akin to a digital vending machine \cite{nickszabosmartcontracts2}.

The Ethereum blockchain is one of the first widely successful and adopted execution environment for 
smart contracts. Due to the decentralised and tamper-resistant properties of Ethereum, 
smart contracts deployed on Ethereum can operate as autonomous actors \cite{smartcontractsforiot},
whose behaviours are completely deterministic and verifiable. Christidis et al also likens them to 
stored procedures in a relational database \cite{smartcontractsforiot}. As an example, consider the
following smart contract for a simple sale of a token X:

\begin{enumerate}
  \item Alice wants to sell, for example, 10 X tokens (in this case they could be non-fungible
  tokens). She writes a buy token function, that accepts 5 Y tokens for each X token sold,
  and then automatically transfers the correct number of X tokens to the buyer's address.
  \item Bob wants to purchase 2 X tokens. He can then submit a transaction calling the 
  buy token function, with 10 Y tokens from his own account balance. The smart contract will
  then execute, and Bob will automatically receive 2 X tokens once the transaction is complete.
  Bob does not need to worry if Alice will honour her part of the contract, because the
  exchange is atomic and coded within the smart contract itself.
  \item Alice can then withdraw the amount of Y tokens she received from the sale, by calling 
  a withdraw profit function. This function would also check that the caller matches Alice's
  preset address, so that no one else is able to steal her profit.
  \item The smart contract can also keep track of the number of X tokens sold, such that 
  sales will revert after the limit of 10 is reached. It can also automatically revert the 
  transaction if the wrong amount of Y tokens are sent, or even refund Bob the correct amount 
  if he sends too many Y tokens.
\end{enumerate}

Smart contracts can therefore be used to describe any arbitrary contractual agreement,
and be used in applications such as voting and governance \cite{ethdao}, 
peer-to-peer marketplaces \cite{opensea}, 
collateralised borrowing platforms \cite{aave} \cite{compound}, or even NFT ticketing \cite{getprotocol} \cite{tkets}.
It is important to note that smart contracts must be completely deterministic in nature, 
or else each node in the decentralised network will output different resulting (but valid) states.

\section{The Ethereum Virtual Machine}

\subsection{Overview}

Smart contracts on Ethereum run in a Turing-complete execution environment called the
Ethereum Virtual Machine (EVM). The EVM runs in a sandboxed environment, and has no access
to the underlying filesystem or network. It is a stack machine that has three types of
storage -- storage, memory, and stack \cite{ethevm}, explained in detail in section \ref{section:storage}.

The EVM runs compiled smart contracts in the form of EVM opcodes \cite{solidityevm}, which can perform
the usual arithmetic, logical and stack operations such as \texttt{XOR}, \texttt{ADD}, and 
\texttt{PUSH}. It also contains blockchain-specific opcodes, such as \texttt{BALANCE}, 
which returns address balance in wei.

An EVM transaction is a message sent from one account to another, and may include any
arbitrary binary data (called the \textit{payload}) and Ether \cite{solidityevm}. If the
target account is a \textit{contract account} (meaning it also stores code deployed on
the blockchain), then that code is executed, and the payload is taken as input.

The EVM also supports message calls, which allows contracts to call other contracts or
send Ether to non-contract accounts \cite{solidityevm}. These are similar to transactions, 
as they each have their own source, target, payload, Ether, gas and return data.

Each transaction in the EVM can generate any number of logs, which are output by each block.
These can be efficiently accessed by off-chain parties, using a bloom filter to find matching
logs of a specified event from each block.

\subsection{Storage}
\label{section:storage}
Every account contains a \texttt{storageRoot}, which is a hash of the root node of a Merkle
Patricia tree that stores the \textit{storage} content of that account \cite{ethereumyellowpaper}.
This consists of a key-value store that maps 256-bit words to 256-bit words \cite{solidityevm}.
This storage type is the most expensive to read, and even more costly to write to, but is the
only storage type that is persisted between transactions. Therefore, developers typically try 
to reduce the amount of storage content used, and often simply store hashes to off-chain storage
solutions such as InterPlanetary File System \cite{ipfs} or Arweave permanent storage \cite{arweave}. A contract
can only access its own storage content.

\textit{Memory} is the second data area where a contract can store non-persistent data. It is
byte-addressable and linear, but reads are limited to a width of 256 bits, while writes
can be either 8-bit or 256-bits wide. \cite{solidityevm} Memory is expanded each time an untouched 256-bit
memory word is accessed, and the corresponding gas cost is paid upfront. This gas cost increases
quadatically as the memory space accessed grows.

The \textit{stack} is the last data area for storing data currently being operated on, since
the EVM is stack-based and not register-based \cite{solidityevm}. This has a maximum
size of 1024 elements of 256-bit words, and is the cheapest of the three types to access in general. 
It is possible to copy one of the topmost 16 elements
to the top of the stack, or swap the topmost element with one of the 16 elements below it.
Other operations (such as \texttt{ADD} or \texttt{SUB}) take the top two elements as input,
and push the result on the top of the stack. It is otherwise not possible to access elements
deeper within the stack without first removing the top elements, or without moving elements
into memory or storage first.

\subsection{Gas consumption}
Gas refers to the "the unit that measures the amount of computational effort required 
to execute specific operations on the Ethereum network" \cite{ethgas}. It is the fee
that is paid in Ether in order to successfully submit and execute a transaction 
on the blockchain. This mechanism is introduced in order to avoid malicious actors
from spamming the network, either by submitting many transactions at once in a denial
of service attack, or by running accidental or intentional infinite loops in smart
contract code. It also incentivises miners to participate in the network, as part of
the gas fees (in the form of tips as introduced in EIP-1559 \cite{eip1559}) is given to the miner .

Each block also has a block limit, which specifies the maximum amount of computation
that can be done within each block. This is currently set at 30 million gas units,
which is 2 times the target block size \cite{ethgas}. Transactions requiring more 
gas than the block limit will therefore always revert, which means that all executions
will either eventually halt, or hit the block limit and revert.

% The amount of gas (in Gwei) paid by each user submitting a transaction is calculated as 
% follows, after the EIP-1559 upgrade:
% \[ \texttt{Gas units} \times (\texttt{Base fee} + \texttt{Tip}) \]
% The gas units represent the amount of gas needed for the computation. The base fee
% represents the base price of each unit of gas, which increases and decreases depending
% on the utilisation of the previous block. The tip represents the amount of Ether the
% user is willing to pay to miners in order to prioritise their transaction. One Gwei is 
% a unit of Ether, equivalent to 0.000000001 Ether.

The schedule for how gas units are calculated is described in detail in Appendix G and H of
the Ethereum Yellow Paper \cite{ethereumyellowpaper}. In summary, each transaction first will require
a base fee of 21000 gas units, which "covers the cost of an elliptic curve operation to recover the 
sender address from the signature as well as the disk and bandwidth space of storing the transaction" 
\cite{ethdesignrationale}. Then, depending on the opcodes of the contract being executed, additional
units of gas will be used up. Most opcodes require a fixed gas unit, or a fixed gas unit per byte of data.
An exception to this is the gas used for memory accesses, where it is calculated as such \cite{ethereumyellowpaper}:
\[ C_{mem}(a) \equiv G_{memory} \cdot a + \left \lfloor{\frac{a^2}{512}}\right \rfloor \]
Here, $G_{memory}$ is the gas unit paid for every every additional word when expanding memory.
$a$ is the number of memory words such that all accesses reference valid memory. Therefore,
memory accesses are only linear up to 724B, after which it becomes quadratic and each memory
expansion costs more.

Another exception would be the \texttt{SSTORE} and \texttt{SELFDESTRUCT} instructions. For
these instructions, the schedule is defined as follows \cite{ethereumyellowpaper}:

\setlength{\tabulinesep}{3pt}
\begin{tabu} to \textwidth{|X[1]|X[1]|X[5]|}
  \toprule
  Name & Value & Description \\
  \midrule
  $G_{\mathrm{sset}}$ & 20000 & Paid for an {\small SSTORE} operation when the value of the storage bit is set to non-zero from zero. \\
  $G_{\mathrm{sreset}}$ & 2900 & Paid for an {\small SSTORE} operation when the value of the storage bit's zeroness remains unchanged or\\
  &&is set to zero. \\
  $R_{\mathrm{sclear}}$ & 15000 & Refund given (added into a \textit{refund counter}) when the storage value is set to zero from\\
  &&non-zero.\\
  $G_{\mathrm{selfdestruct}}$ & 5000 & Amount of gas to pay for a {\small SELFDESTRUCT} operation. \\
  $R_{\mathrm{selfdestruct}}$ & 24000 & Refund given (added into a \textit{refund counter}) for self-destructing an account. \\
  \bottomrule
\end{tabu}\\

The refund counter tracks the units of gas that is refunded to the user upon successful
completion of a transaction, which means that the transaction does not revert or invoke an
out-of-gas exception. Only up to half of the total gas used by the transaction can be refunded
using the refund counter. This mechanism is introduced to encourage freeing up storage resources
used by the Ethereum blockchain.

Therefore, we can see that the estimation of gas usage for any given function is not trivial
or easily predictable, and often depends on the current state of the blockchain, as well as the 
input parameters given. It would also require detailed modelling of the gas and refund counter, 
as well as other internals of the EVM implementation.

\section{Solidity}
\subsection{The Solidity language}

Solidity is a statically-typed, object-oriented, high-level programming language for developing
Ethereum smart contracts \cite{ethsolidity}. It is a language in active development with
numerous breaking changes in each major release, all of which are documented extensively online 
\cite{soliditydocs}. We will summarise some of its notable features in this section.

\textbf{Inheritance.} Solidity supports inheritance, by extending other contracts. These contracts can be interfaces (\textit{abstract contracts}),
with incomplete implementations of function signatures. A prime example of this would be the ERC-20
and ERC-721 interfaces \cite{etherc20} \cite{etherc721}, which are standard interfaces that Ethereum 
tokens and non-fungible tokens respectively are expected to follow. 

\textbf{Libraries.} Solidity also supports the
use of libraries, which can contain re-usable functions or structs that are referenced by other
contracts. OpenZeppelin for example has a wide range of utility libraries for implementing enumerable sets,
safe math, and so on \cite{openzeppelincontracts}.

\textbf{Function scopes.} Solidity allows developers to define the scopes of each function --
\texttt{internal}, \texttt{external}, \texttt{private}, or \texttt{public} \cite{soliditydocsscope}. 
\texttt{internal} functions can only be called by the current contract or any contracts that
extend it, and are translated into simple jumps within the EVM. The current memory is not cleared,
making such function calls very efficient. \texttt{private} functions form a subset of
\texttt{internal} functions, in that it is stricter by only being visible to the current contract.
\texttt{external} functions can only be called via transactions or message calls,
and all arguments must be copied into memory first. This means that they can only be called
"internally" by the same contract via the \texttt{this} keyword, which effectively makes an 
external message call to itself. \texttt{public} functions are a superset of \texttt{external} functions,
in that they can be called internally or externally.

\textbf{Storage types.} Solidity supports the explicit definition of where variables are stored -- 
\texttt{storage}, \texttt{memory}, \texttt{calldata}, or \texttt{stack}. \texttt{calldata} is
an immutable, non-persistent area for storing \textit{function arguments} only, and is
reccomended to be used whenever possible as it avoids copies or modification of data.
\texttt{storage} and \texttt{memory} stores the variable in the respective data areas, 
as described in section \ref{section:storage}. 


\subsection{The Solidity compiler and EVM assembly}

The Solidity compiler translates the high-level Solidity code into low-level EVM bytecode,
as well as generating other metadata such as the contract Application Binary Interface (ABI),
and coarse gas estimates for each function. It supports a built-in optimiser, which
takes an \texttt{--optimize-runs} parameter. This parameter balances the gas costs of the
initial deployment, with the gas costs of future function calls. For example, for
a contract that would be called many times after it is deployed, one can set \texttt{--optimize-runs=200},
which will generate a larger bytecode, but make future transactions cheaper. The difference
that this setting makes in the bytecode may be interesting for developers to examine.

To illustrate the operation of the compiler, let us examine the following example contract,
as well as the (truncated) output from the Solidity compiler.\\

\begin{lstlisting}[language=Solidity, caption={An example Solidity contract}, label={lst:solcsolidityexp}, basicstyle=\ttfamily\scriptsize]
pragma solidity >=0.5.0 <0.9.0;
contract C {
    function one() public pure returns (uint) {
        return 1;
    }
}
\end{lstlisting}

\begin{lstlisting}[language=plantuml, caption={EVM assembly from the Solidity compiler}, label={lst:solcexp}, basicstyle=\ttfamily\scriptsize]
======= test.sol:C =======
...
sub_0: assembly {
        /* "test.sol":33:123  contract C {... */
      mstore(0x40, 0x80)
      callvalue
...
    tag_1:
      pop
      jumpi(tag_2, lt(calldatasize, 0x04))
      shr(0xe0, calldataload(0x00))
...
    tag_5:
        /* "test.sol":87:91  uint */
      0x00
        /* "test.sol":111:112  1 */
      0x01
        /* "test.sol":104:112  return 1 */
      swap1
      pop
        /* "test.sol":51:120  function one() public pure returns (uint) {... */
      swap1
      jump	// out
        /* "#utility.yul":7:84   */
...
        /* "#utility.yul":7:84   */
    tag_9:
        /* "#utility.yul":44:51   */
      0x00
        /* "#utility.yul":73:78   */
      dup2
        /* "#utility.yul":62:78   */
...

    auxdata: [...]
}
\end{lstlisting}

As seen in Listing \ref{lst:solcexp}, the Solidity code is translated into subroutines with
multiple tags as jump destinations, similar to the GCC compiler. It also includes control flow
instructions such as \texttt{JUMPI} and \texttt{JUMP}. Debug information is also
embedded into the output, which describes the code fragment that a particular segment of
assembly maps to, similar to the \texttt{-g} flag in the GCC compiler. These are in the 
form of character offsets from the top of the source file, rather than line and character 
numbers. 

A detailed explanation of how the EVM assembly maps to the Solidity instructions
is included in section \ref{section:Yul}, where we examine this in relation to the Yul
intermediate representation.

An interesting feature is the \texttt{\#utility.yul} file that is referenced throughout
the output. This file contains extra Yul code (see Section \ref{section:Yul}) that are
generated automatically by the compiler, and its addition is triggered by the usage of
specific language features, such as the ABI encoder for function paramters and return values.
They are not specific to the user's source code, but are likely to be executed in control flow
jumps from other parts of user code.

The Solidity compiler also has other command-line features as documented online 
\cite{solcinputoptions}, such as the manual linking of library addresses, setting of 
optimisation options like the inliner or the deduplicator, as well as generation of
additional files such as the Abstract Syntax Tree (AST), which can be very useful for the 
construction of control flow graphs. A recent addition to this
is the (experimental) function to generate an intermediate representation of the user Solidity code in
Yul, which we will elaborate more in the next section.

\subsection{Yul}
\label{section:Yul}

Yul is an intermediate language that is designed to be compiled into bytecode for different
backends \cite{solcyul}, such as future EVM versions and EWASM, and acts as a common denominator
for all current and future platforms. Yul is also designed to be human readable, with high level constructs
such as \texttt{for} and \texttt{switch}, but still suitable for whole-program optimisation.

As an example, we examine the Yul intermediate representation (IR) output from the Solidity compiler,
using the same Solidity code in Listing \ref{lst:solcsolidityexp}:\\

\begin{lstlisting}[language=plantuml, caption={Yul IR from the Solidity compiler}, label={lst:solcexp}, basicstyle=\ttfamily\scriptsize]
 /// @use-src 0:"test.sol"
 object "C_10" {
     ...
     /// @use-src 0:"test.sol"
     object "C_10_deployed" {
         code {
             ...
             /// @ast-id 9
             /// @src 0:51:120  "function one() public pure returns (uint) {..."
             function fun_one_9() -> var__4 {
                 /// @src 0:87:91  "uint"
                 let zero_t_uint256_1 := zero_value_for_split_t_uint256()
                 var__4 := zero_t_uint256_1
 
                 /// @src 0:111:112  "1"
                 let expr_6 := 0x01
                 /// @src 0:104:112  "return 1"
                 var__4 := convert_t_rational_1_by_1_to_t_uint256(expr_6)
                 leave
 
             }
             /// @src 0:33:123  "contract C {..."
 
         }
        ...
     }
 }
\end{lstlisting}

We see that the Yul code far more resembles the original Solidity code. Here, we also
observe how the stack is being used in our simple \texttt{one()} function:

\begin{enumerate}
  \item First, a temporary local variable \texttt{zero\_t\_uint256\_1} is pushed onto the stack with value \texttt{0x00}. 
  Then, it is assigned to the return variable \texttt{var\_\_4}. In EVM assembly, this maps to
  the \texttt{0x00} instruction on line 15 of Listing \ref{lst:solcexp}.
  \item Next, another temporary local variable \texttt{expr\_6} is pushed onto the stack with
  value \texttt{0x01}. It is then assigned to the return variable \texttt{var\_\_4} again, and the function
  exits, returning the value in \texttt{var\_\_4}. In EVM assembly, this maps to the \texttt{0x01}
  instruction on line 17. Then, to assign the new value to \texttt{var\_\_4}, we first swap it with the
  old value (currently on the 1st slot in the stack below \texttt{0x01}), on line 19. Finally, we pop the
  old value off the stack, and swap the next jump instruction address onto the top of the stack, in
  lines 20 and 22.
\end{enumerate}

Similar to the EVM assembly, the generated Yul code also contains debug information about
the mapping to the original source code. This can be very useful for understanding how the
final EVM assembly operates, as seen in our previous example. As such, it would be
interesting to generate a similar mapping for this project to the Yul intermediate
representation, in addition to the final EVM assembly.

