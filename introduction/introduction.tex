\chapter{Introduction}

% \begin{figure}[tb]
% \centering
% \includegraphics[width = 0.4\hsize]{./figures/imperial}
% \caption{Imperial College Logo. It's nice blue, and the font is quite stylish. But you can choose a different one if you don't like it.}
% \label{fig:logo}
% \end{figure}
\begin{chapquote}{Vitalik Buterin \cite{vitalikquote}}
    Whereas most technologies tend to automate workers on the 
    periphery doing menial tasks, blockchains automate away the center.
    Instead of putting the taxi driver out of a job, blockchain puts 
    Uber out of a job and lets the taxi drivers work with the customer 
    directly.
\end{chapquote}

\section{Overview}
In recent years, Ethereum has emerged as a promising network
for building next-generation, decentralised applications (DApps). Ethereum makes use of
blockchain technology to achieve consensus in a distributed, public,
and immutable ledger. It supports the running of arbitrary code, called smart contracts,
in its own execution environment, called the Ethereum Virtual Machine (EVM).
These smart contracts are typically written in a high level programming
language, most notably Solidity, and are often used to power the transactional
backends of various DApps. They are also used to define the behaviour of utility tokens 
or various non-fungible tokens (NFTs), which are simply smart contracts that extend a standard
interface. Examples would be decentralised exchanges such 
as Uniswap \cite{uniswap} and Sushiswap, decentralised borrowing and lending platforms such as
Aave \cite{aave} and Compound \cite{compound}, algorithmic stablecoins such as Dai \cite{makerdao} and TerraUSD \cite{terra}, and NFT-powered gaming 
ecosystems such as Axie Infinity \cite{axieinfinity} and the Sandbox \cite{sandbox}.

Part of the appeal and increasing usage of smart contracts is that it allows transactions and exchanges
to take place in a trustless, non-custodial, and cryptographically secure fashion.
For example, in the case of a decentralised exchange, there is no need for two
untrusted parties to hand over custody of their assets to a trusted intermediary,
who would then conduct the actual exchange. Smart contracts are instead able to conduct
the exchange of tokens or assets in a single, atomic transaction,
which will either succeed, or revert and leave both parties unaffected.
This powerful ability is imparted by Ethereum's immutable and decentralised nature, 
as no single party can control or alter the state of the blockchain directly.

The result is an explosion in smart contract development and total
value locked (TVL), from around \$600 million to over \$150 billion in the
span of two years \cite{defillama}. However, this 
has also brought on scalability issues that limit the overall usefulness
of the Ethereum network. In simple terms, smart contract transactions 
each require a certain amount of ``gas'' in order to be processed, 
depending on its complexity. Users bid for the price of gas they are
willing to pay, in ether, and miners typically choose to process the 
transactions with higher bids. Since only a limited number of transactions
can be processed in each block, this results in an efficient market
economy between miners and users. 

However, when more users start utilising the network, the gas prices
naturally start rising as demand grows. Lately, due to the boom in decentralised 
finance (DeFi) platforms and NFTs, the gas price has 
surged by almost 10 times its price compared to two years ago \cite{gashistory}. 
Transactions are costing up to \$63 on average in November \cite{gasfee1}, and this has made the
Ethereum blockchain inhibitive to use for most users \cite{gasfee2}.  
Uniswap, a popular decentralised exchange for trading tokens on the Ethereum network,
has spent more than \$8.6 million per day in gas fees alone, as reported last November \cite{gasusageperday}.
Therefore, there is a clear need not only for better scaling of the Ethereum blockchain, but 
also for developers creating DApps on the Ethereum blockchain to optimise their 
smart contract code for gas usage.

Despite the enormous cost savings that can be achieved via more gas-efficient smart contracts, 
most of the development on Ethereum smart contract analysis currently
surrounds verification of smart contracts rather than gas optimisation \cite{smartcontracttools}.
The Solidity compiler itself does offer an estimate for gas usage \cite{soliditydocs}, although only at a 
course-grained function-call level, making analysis of code within each function difficult.
It also only performs a rudimentary best-effort estimate, and outputs infinity whenever the
function becomes too complex. Given that Solidity is a language under active development
with numerous breaking changes implemented at each major version release \cite{solidity07breaking} \cite{solidity08breaking}, the few tools
designed with gas estimation in mind are also either no longer maintained or broken, or
inadequate for fine-grained, line-level gas analysis. We will discuss some of these existing
tools in the background section later.

Currently, most Solidity developers simply make use of general tools such as the Remix IDE, Truffle and Ganache for
smart contract development \cite{consensystools}. Such tools are not specialised for detailed gas usage profiling, and developers
may potentially miss out on optimisations that may save up to thousands of dollars worth of gas \cite{gaschecker},
or worse, possibly render a contract call unusable due to reaching the gas limits of each block \cite{governmentalstuck}.
Gas optimisation is a step that every developer has to carefully consider when deploying an immutable contract on the blockchain,
and therefore creating a tool that calculates the gas usage for each bytecode instruction 
and line of code could greatly enhance and simplify such analysis, not only for 
smart contracts on the Ethereum blockchain, but also for all EVM compatible chains such as
the Binance Smart Chain, Polygon, and the Avalanche Contract Chain \cite{evmcompatiblechains}.

\section{Planned contributions}

With the goal of gas optimisation in mind, this project aims to improve on current 
smart contract development tools by creating a smart contract compiler explorer tool,
which can also display a heatmap of the gas usage of each bytecode instruction, while mapping
them to the corresponding lines in the user's source code. This is inspired by Godbolt \cite{godbolt},
a popular compiler explorer tool for C++ for finding optimisations in the GCC compiler.
The following summarises the intended contributions this project plans to achieve.

\begin{enumerate}
    \item \textbf{Compilation and mapping of each line of user Solidity code into their corresponding EVM bytecode and Yul intermediate representation.}
    
      Similar to Godbolt which is able to easily map and visualise which assembly
      instructions correspond to its high-level C++ code, we want to achieve a
      similar result with EVM assembly bytecode and its corresponding Solidity code.
      We also want to be able to generate the Yul intermediate representation for
      the Solidity code, which is a recently introduced language between Solidity
      and EVM assembly.
    \item \textbf{Analysis of gas consumption via symbolic execution for each EVM instruction}
    
      The schedule for gas consumption of EVM instructions is carefully laid out 
      in the Ethereum Yellow Paper \cite{ethereumyellowpaper}. Using this,
      we want to conduct an analysis of gas consumption for each line
      of Solidity code, instead of simply at a function call level. To do this, we 
      propose using symbolic execution to traverse each possible execution path, 
      and then aggregating the gas costs for each basic block. 
      To address the problem of path explosion, we also propose the implementation of a
      non-iterative loop bounds analysis that can generate a parametric loop bound using
      polytopes. Then, we can allow users to define these external parameters to obtain a 
      more precise estimate.

    \item \textbf{Generation of heatmaps of gas usage per line of code}
    
      To make visualising gas usage simpler, we plan on generating a heatmap on top of
      the direct mapping from EVM bytecode instructions to Solidity code, which would 
      display at a glance which lines of code are consuming the most amount of gas, and 
      may potentially have room for optimisation.

    % \item \textbf{Automatic identification of gas inefficient patterns}
      
    %   We plan on also identifying some common gas-hungry code patterns, 
    %   based on previous literature on this topic \cite{gaschecker} \cite{designpatternsforgasoptimisation} \cite{underoptimisedcontractsdevour}, as well as our 
    %   own static analysis of the gas usage, and provide suggestions on how 
    %   to improve them.
      
\end{enumerate}


\section{Outline of report}

In Chapter \ref{chap:background}, we provide a summary of smart contracts on Ethereum, the internals of the
Ethereum Virtual Machine and its gas schedule, as well as a background on Solidity and the 
Yul intermediate representation. In Chapter \ref{chap:gasbounds}, we describe our proposed
implementation to estimate gas costs using symbolic execution, and a non-iterative algorithm using
polytopes to statically infer loop bounds. In Chapter \ref{chap:related}, we briefly examine two 
related projects, as well as their results and limitations. In Chapter \ref{chap:eval}, we 
propose the method we intend to measure the project's quality and effectiveness by. In 
Chapter \ref{chap:ethics}, we examine the possible ethical issues that may arise from 
this project. Finally, in Chapter \ref{chap:projplan}, we discuss the proposed timeline 
for the completion of this project, including stretch goals and fallback goals under various
circumstances.