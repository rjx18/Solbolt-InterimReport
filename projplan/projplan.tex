\chapter{Project Plan}
\label{chap:projplan}

As of the writing of this report, the following foundational steps have been
completed:
\begin{itemize}
  \item Testing of the Solidity compiler for its capabilities and scope of debug logs emitted
  \item Testing of various symbolic execution engines (Mythril and Oyente) to investigate their capabilities
  \item Creation of a basic web application that is able to decompile a Solidity contract using the javascript Solidity compiler
  \item Research on previous gas estimation techniques for EVM, as well as various loop bounds analysis methods for symbolic execution
\end{itemize}

Moving forward, the plan for the project is as follows:
\begin{enumerate}
  \item Building the compilation pipeline for the basic web app that automatically compiles Solidity code into both EVM and Yul IR. \textbf{Suggested timeline: 1.5 weeks (31 Jan - 9 Feb)}
  \item Development of the mapping feature that allows compiled EVM and Yul code to be mapped to the corresponding Solidity code. \textbf{Suggested timeline: 1.5 weeks (10 Feb - 21 Feb)}
  \item Extending the Mythril symbolic execution engine to provide aggregate gas estimates for each basic block. \textbf{Suggested timeline: 2 weeks (22 Feb - 7 Mar)}
  \item Development of the heatmap feature that takes the gas estimates calculated and displays it in a heat map, on top of the compiled EVM and Yul IR. \textbf{Suggested timeline: 2 weeks (7 Mar - 21 Mar)}
  \item Extension of the Solidity Yul compiler with our loop analysis algorithm, to emit parametric loop bounds within the output EVM. \textbf{Suggested timeline: 3 weeks (21 Mar - 11 Apr)}
  \item Extension of web app to take into account the emitted parametric loop bounds during gas estimate calculation. \textbf{Suggested timeline: 1.5 weeks (14 Apr - 21 Apr)}
  \item \textit{(Stretch)} Development of a symbolic debugger using the Mythril engine, that allows the user to step through an execution and examine the values within the stack, the programme counter, and the gas used. \textbf{Suggested timeline: 4 weeks}
  \item \textit{(Fallback)} Development of a feature that builds and displays the control flow graph for both Yul and EVM bytecode \textbf{Suggested timeline: 2 weeks}
  \item Conduct evaluation on the web application, and collect user feedback \textbf{Suggested timeline: 2 weeks (10 May - 24 May)}
  \item Writing of draft report, and meet with supervisor \textbf{Suggested timeline: 2 weeks (24 May - 7 Jun)}
  \item Finalise report and create and rehearse for final presentation \textbf{Suggested timeline: 2 weeks (7 Jun - 20 Jun)}
\end{enumerate}

For items 5 and 6, if it is found to be unable to achieve the stated goal in time, or if the previous
steps take longer than expected, the project will fallback into developing step 8 instead, which
should be more manageable to complete. If the project goes very well and there is extra time left,
we will focus on developing step 7, which would be an additional (but useful) feature for the tool.
We have also included 2 weeks of buffer time for courseworks and examinations.